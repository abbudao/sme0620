\chapter{Principais Modelos Probabilísticos}
 \section{Modelos Discretos}
 \subsection{Distribuição Uniforme Discreta}
 \begin{description}
   \item  A v.a.\ assume cada um de seus valores com igual probabilidade.

   \item [Definição:] A v.a.\ discreta $X$, assumindo valores $x_1,x_2,\dots,x_k$ tem 
 distribuição uniforme se, e somente se, sua f.m.p.\ é definida por: 
 \begin{align}
   p(x)=P(X=x)=\begin{cases}
     \frac{1}{k}, \text{\quad} x=x_1,x_2,\dots,x_k\\
     0, \text{\quad caso contrário}
   \end{cases}
 \end{align}
\item  [Notação:] $X \sim U_{d}(k)$
 A média e variância de x é:
 \begin{align}
   \mu_{x}=E(x)=\sum_{x \in R_{x}}x_i P(X=x_i)\\ \nonumber
   = \sum_{x \in R_{x}}x_i \frac{1}{k}= \frac{1}{k}\sum_{x_i \in R_{x}} x_i\\
   Var(x)=E(X^2)-\left(E(X)\right)^2\\ \nonumber
   = \sum_{x_i \in R_{x}}\left(x_i -\mu_x \right)^2 \times p\left(X=x_i \right)\\  
   =\frac{1}{k} \left(\sum^k_{i=1} x_i^2 - \frac{(\sum_{i=1}^k)^2}{k}\right) \nonumber
 \end{align}
\item [Exemplo:] Seja $x$ uma v.a.\ que indica o $n^o$ de pontos marcados na face 
 superior de um dado quando ele é lançado. Portanto, temos uma distribuição 
 uniforme discreta, cuja f.m.p.\ é dada por: 
 \begin{align}
   p(x)=P(X=x)-\begin{cases}
     \frac{1}{6},\text{\quad} x=1,2,3,4,5,6\\
     0, \text{\quad caso contrário}
   \end{cases}
 \end{align}
 E o $E(x)$ é: 
 \begin{align}
   \mu_{x}=E(x)=\frac{1}{6}(1+2+3+4+5+6)\\
   =\frac{1}{6}21= \frac{7}{2}=3,5
 \end{align}
 E a $Var(x)$ é:
 \begin{align}
   E(x^2)=\frac{1}{6}(1^2 + 2^2 + \dots + 6^2)\\
   =\frac{91}{6}=15,17\\
   var(x)=\frac{91}{6}- \{\frac{7}{2}\}^2=\frac{105}{36}=2,92
 \end{align}
 \end{description}
 \subsection{Distribuição Bernoulli}
 \begin{description}
   \item Considere uma v.a.\ $x$ que assume apenas dois valores 1, se ocorrer sucesso,$0$, se ocorrer fracasso. Indicaremos por $p$ a probabilidade de sucesso, isto é $P(\text{Sucesso})=P(x=1)=p$.

\item [Definição:] A v.a.\ discreta $x$ que assume apenas valores 0 ou 1 tem Distribuição
 Bernoulli se, somente se, sua f.m.p.\ é definida por:
 \begin{align}
   p(x)=P(X=x)= \begin{cases}
     p^{x}(1-p)^{1-x}, \text{\quad} x=0,1, \text{\quad} 0<p<1\\
     0, \text{\quad caso contrário}
   \end{cases}
 \end{align}
\item[ Notação:] $X \sim Ber(p)$

\item [Exemplos:]
  $$p(x)=\left(\frac{2}{5}\right)^x\left(1-\frac{2}{5}\right)^{1-x} \text{\quad},x=0,1$$\\
 $$p(x)=\left(\frac{7}{8}\right)^{x}\left(1-\frac{7}{8}\right)^{1-x} \text{\quad }, x=0,1$$

 De forma similar, podemos apresentar $p(x)$ por:
 \begin{figure} [H]
   \centering
   \begin{tabular}{c c c}
     \toprule
     x&0&1\\ \midrule
     $P(X=x)$&$1-p$&$p$\\ \bottomrule
   \end{tabular}
   \label{fig:7}
 \end{figure}
 A média de $x$ é: 
 \begin{align}
   \mu_{x}=E(x)=p
 \end{align}
 e a variância de $x$ é: 
 \begin{align}
   \sigma^2 = var(x)=p (1-p)
 \end{align}
 \item [Exemplo:] Suponha o lançamento de um dado perfeito e o interesse é ocorrer a face
 3.
 \begin{align*}
   X=\begin{cases}
     1, \text{\quad se ocorrer a face }3\\
     0, \text{\quad caso contrário}\{1,2,4,5,6\}]
   \end{cases}
 \end{align*}
 Cuja f.m.p.\ é dada por:
 \begin{align}
   p(x)=P(X=x)=\begin{cases}
     \frac{1}{6}(1-\frac{1}{6})^{1-x}, \text{\quad}x=0,1\\
     0, \text{\quad caso contrário}
   \end{cases}
 \end{align}
 \begin{figure} 
   \centering
   \begin{tabular}{c c c}
     \toprule
     x&0&1\\ \midrule
     $P(X=x)$&$\frac{5}{6}$&$\frac{1}{6}$\\ \bottomrule
   \end{tabular}
   \label{fig:8}
 \end{figure}
\begin{align*}
   X \sim Ber(\frac{1}{6})
\end{align*}

 A média de $x$ é: 
 \begin{align*}
   \mu_{x}=(X)=p=\frac{1}{6}
 \end{align*}

 E a variância de $x$ é: 
 \begin{align*}
   \sigma^2= var(x)=p(1-p)=\frac{1}{6} \times \frac{5}{6}=\frac{5}{36}
 \end{align*}
 \end{description}
 \subsection{Distribuição Binominal}

 \begin{description}
   \item  Considere $M$ ensaios de Bernoulli independentes, todos com a mesma probabilidade
 de sucesso $p$. A v.a.\ que conta o número de sucessos nos $M$ ensaios de Bernoulli
 é denominada v.a.\ binomial com parâmetro $M$ e $p$.

   \item[Definição:] A v.a.\ discreta $x$, correspondente ao $n^o$ de sucesso em $M$ ensaios
     independentes de Bernoulli, tem distribuição binomial se, e somente se, sua 
     f.m.p.\ é definida por: 
     \begin{align}
       p(x)=P(X=x)=\begin{cases}
         \binom{M}{x}p^x (1-p)^{M-x}, \text{\quad}x=0,1,\dots,M  \quad  ,0<p<1\\
         0, \text{\quad Caso contrário}
       \end{cases}
     \end{align}

   \item{Notação}:$ X \sim Bin(M,p)\\$
   \item     [Obs:] $\binom{M}{x}=\frac{M!}{x!(M-x)!}$

     Temos que: 
     \begin{align}
       \mu_{x}=E(X)=Mp \\
       \sigma^2 =var(x)=Mp(1-p)
     \end{align}
   \item[Exemplo]: Considere uma linha de produção, onde 3 peças são selecionadas aleatoriamente
     e são classificadas como defeituosas (D) ou não-defeituosas(N).\\
     $X_{1},X_{2},X_{3}$ são variáveis aleatórias que assumem $1$ se a peça for não-defeituosa e $0$ caso 
     contrário. A probabilidade da peça ser não-defeituosa é $p$ e , consequentemente, 
     a probabilidade da peça ser defeituosa é $1-p$. Estamos interessados na distribuição
     de:$Y=X_{1}+X_{2}+X_{3}$
     \begin{align*}
       X_1 = \begin{cases}
         1, \quad \text{Peça não deituosa}\\
         0, \quad \text{Peça deituosa}
       \end{cases}
     \end{align*}
     \begin{figure}[H]
       \centering
       % Set the overall layout of the tree
\tikzstyle{level 1}=[level distance=3.5cm, sibling distance=3.5cm]
\tikzstyle{level 2}=[level distance=3.5cm, sibling distance=2cm]

% Define styles for bags and leafs
\tikzstyle{bag} = [text width=4em, text centered]
\tikzstyle{end} = [circle, minimum width=3pt,fill, inner sep=0pt]

% The sloped option gives rotated edge labels. Personally
% I find sloped labels a bit difficult to read. Remove the sloped options
% to get horizontal labels. 
\begin{tikzpicture}[grow=right, sloped]
  %Create a start Node and childs
  \node[bag] {$ D $}
  child {
    node[bag]{$ N $}
    child {
      node[end,label=right: {$N$}]{}
      edge from parent
    }
    child {
      node[end,label=right: {$D$}]{}
      edge from parent
    }
    edge from parent
  }
  child {
    node[bag]{$ D $} 
   child {
      node[end,label=right: {$N$}]{}
      edge from parent
    }
    child {
      node[end,label=right: {$D$}]{}
      edge from parent
    }
    edge from parent
  }
  ;
\end{tikzpicture}

       \label{fig:}
     \end{figure}
     \begin{figure}[H]
     \begin{tabular}{c c c c c c }
       \toprule
       & $X_1$  & $X_2$ & $X_3$   & $Y= X_1 +X_2+ X_3$ & $ P(Y=y)$\\ \midrule
       DNN &0 & 1 & 1 & 2 & $p^2(1-p)$\\\midrule

       DND& 0 & 1 & 0 & 1 & $p(1-p)^2$\\\midrule

       DDN& 0 & 0 & 1 & 1 & $p(1-p)^2$\\\midrule

       DDD&  0 & 0 & 0 & 0 & $(1-p)^3$\\\midrule

       NNN&   1 & 1 & 1 & 3 & $p^3$\\\midrule

       NND& 1 & 1 & 0 & 2 & $p^2(1-p)$\\\midrule

       NDN& 1 & 0 & 1 & 2 & $p^2(1-p)$\\\midrule

       NDD& 1 & 0 & 0 & 1 & $p(1-p)^2$ \\ \bottomrule

     \end{tabular}
   \end{figure}

     Abaixo estão os possíveis resultados do experimento:

     %Arvore zica aqui, pegar de algum colega

     \begin{figure}[H]
       \centering
       \begin{tabular}{c c c c c}
         \toprule
         y&0&1&2&3\\ \midrule
         $P(Y=y)$&$(1-p)^3$&$3p(1-p)^2$&$3p^2(1-p)$&$p^3$\\ \bottomrule
       \end{tabular}
       \label{fig:127}
     \end{figure}
      Ou seja,
      \begin{align*}
        p(y)=P(Y=y)=\begin{cases}
          \binom{3}{y}p^y (1-p)^{3-y}, \text{\quad}y=0,1,2,3\\
          0, \text{\quad Caso contrário}
        \end{cases}
      \end{align*}
    \item[Exemplo]:
      
      Uma rede varejista compra certo tipo de equipamento eletrônico. O 
      fabricante indica que a taxa de quipamentos em perfeito estado é $97\%$.
      \begin{enumerate}[label=(\alph*)]
        \item Seleciona-se ao acaso 20 itens. Qual a probabilidade de haver pelo menos 
          um item defeituoso nesses 20?
        \item Seleciona-se aleatoriamente 20 itens em cada 10 carregamentos, qual a 
          probabilidade de haver 3 carregamentos com pelo menos um item defeituoso?
      \end{enumerate}
      $p$ taxa de equipamentos em perfeito estado\\
      $$p=0,97$$\\
      $y$: $n^o$ de equipamentos em perfeito estado.\\
      $$y \sim Binom(20;0,97)$$\\

    \item[Resolução]:

      \begin{enumerate}[label=(\alph*)]

        \item    
          \begin{align*}
            M=\quad \text{20 itens}\\
            p: \quad \text{Probabilidade de pelo menos um item defeituoso.}\\
          p=0,03
        \end{align*}
          \begin{align*}
           X_i= \begin{cases}
             1, \quad \text{se o equipamente e defeituoso}\\
             0, \quad \text{c.c.\ }
            \end{cases} 
          \end{align*}
          \begin{align*}
X           \sim Binom(20:0,03) \quad x=1,2,3,4,5,\dots,20
         \end{align*}
          \begin{align*}
           Y=X_1 + X_2 + \ldots +X_20 
          \end{align*}
          \begin{align*}
            P(Y\ge 1)= 1-P(Y<1)\\
            =1-P(Y=0)
          \end{align*}
          \begin{align*}
            P(Y=0)= \binom{20}{0} (0,03)^0 (1-0,03)^{20-0}
            =0,5438
          \end{align*}
          \begin{align*}
            P(Y\ge 1)= 1-0,5438\\
            =0,4562
          \end{align*}
          Ou, resolvendo pela v.a.\ $Y$, temos:
\begin{align*}
           X_i= \begin{cases}
             1, \quad \text{se o equipamento for perfeito} \\
             0, \quad \text{se o equipamente for defeituoso}
            \end{cases} 
          \end{align*}
          \begin{align*}
            P(Y\le 19)=1-P(Y>19)\\
            =1-P(Y=20)
          \end{align*}
          \begin{align*}
            P(y=20)=\binom{20}{20}0,97^{20}(1-0,97)^{20-20}\\
            =0,5438
          \end{align*}
          \begin{align*}
            P(y\le 19)=1-0,5339\\
            =0,4562
          \end{align*}
        \item
          \begin{align*}
            10\text{ Carregamentos}\\
           20  \text{ itens são selecionados} \\
           Y: \text{$n^o$ de carregamento com pelo menos um item defeituoso} \\
          M=10\\
          Y=0,1,2,\ldots,10\\
          p:\quad \text{Proporção de ter pelo menos um item defeituoso em um carregamento}\\
          p=0,4562\\
          Y \sim B(10:0,4562)
        \end{align*}
        \begin{align*}
            P(Y=3)=\binom{10}{3}(0,4562)^3 (1-0,4562)^{10-3}  =0,1602
          \end{align*}
      \end{enumerate}
    \end{description}
     % \emph{Exercício}: Um fabricante adquire certo tipo de componente de um fornecedor.
     % Segundo este fornecedor, a proporção de componentes defeituoso é de 2\%. O fabricante
     % adquire 10 lotes por mês e de cada lote são selecionados 15 componentes para 
     % inspeção. Qual a probabilidade de que sejam encontrados 3 lotes com mais de um 
     % componente defeituoso?

      \subsection{Distribuição Geométrica}
      \begin{description}
        \item  Considere uma sequência de ensaios Bernoulli independentes em probabilidade de 
      sucesso $p$ ($0<p<1$). Seja a v.a.\ $x$ o $n^o$ de fracassos até a ocorrência 
      do $1^o$ sucesso. Similarmente, a v.a.\ $x$ pode ser vista como o $n^o$ de ensaios que precedem 
      o primeiro sucesso.

    \item[Definição]: A v.a.\ $x$ tem distribuição geométrica se, e somente se, sua 
      f.m.p.\ é definida por: 
      \begin{align}
        p_{x}(x)=P(X=x)=\begin{cases}
          p(1-p)^x, \quad x=0,1,2,\dots \quad 0<p<1\\
          0, \quad \text{caso contrário}
        \end{cases}
      \end{align}
      \begin{center}Notação: $X \sim Geo(p)$\end{center}
      Temos que:
      \begin{align}
        \mu_{x}=E(x)=\frac{1-p}{p} \text{\quad e a }\\
        \sigma^2 = var(x)=\frac{1-p}{p^2}
      \end{align}
      Similarmente, a variável aleatória $Y$ pode ser vista como o $n^o$ de ensaios que
      precedem o primeiro sucesso assim, $Y$ tem a distribuição geométrica com f.m.p.\ dada por:
        \begin{align}
        p_y(y)=P(Y>y)=\begin{cases}
          (1-p)^{y-1} \text{\quad }p,y=1,2,\dots \\
          0, \text{\quad Caso contrário}
        \end{cases}
      \end{align}
      Neste caso, a v.a.\ $Y$ pode ser vista como $Y=X+1$ e, consequentemente, o valor esperado de y é:
      \begin{align} 
      mu_y=E(Y)=E(X+1)=E(X)=E(X)+1= \frac{1}{p}
    \end{align}
    E a variância de $Y$ e:
    \begin{align}
      \sigma_y^2=var(Y)=var(X+1)=var(X)= \frac{1-p}{p}^2
    \end{align}
  \item[Exemplo]: Um pesquisador está realizando um experimentos químicos independetes 
      e sabe que a probabilidade de que cada cada experimento apresnete uma reação 
      positiva é $0,3$. QUal é a probabilidade de que menos de 3 reações negativas 
      ocorram antes da primeira positiva?

      $x$:$n^o$ de ocorrência positiva 
      \begin{align*}
        x=&\begin{cases}
          1, \text{\quad se a reação positiva}\\
          0, \text{\quad caso contrário}
        \end{cases}
      \end{align*}
      $$P(x=1)=p=0,3$$\\
      $X: n^o$ de reações negativas até a ocorrência de uma positiva\\
      \begin{center}$Y \sim Geo(0,3)$\end{center}
      \begin{align*}
        P(X<3)=P(X=0)+P(X=1)+P(X=2)\\
        P(X=0)=(1-0,3)^0 \times 0,3 = 0,3\\
        P(X=1)=(1-0,3)^1 \times 0,3 = 0,21\\
        P(X=2)=(1-0,3)^2 \times 0,3 = 0,147\\
        P(X<3)=0,3+0,21+0147=0,657
      \end{align*}
      Ou ainda,
      \begin{align*}
        Y: n^o \quad \text{de ensaios até a ocorrência de reação positiva}\\
        R_y = \{ 1,2,3,\ldots \}\\
        P( Y<4 ) = P (Y \leq 3) = P(Y=1)+ P(Y=2)+P(Y=3)\\
        P(Y=y)= p(1-p)^{y-1} \quad y=1,2,\dots
      \end{align*}
     % Exercício: Pensar no exercício anterior com o $n^o$ de ensaios.
    \end{description}
      \subsection{Distribuição Binomial Negativa}
      \begin{description}
        \item  Considere ensaios independetes de Bernoulli ($p$) e definimos $X$ como o $n^o$ 
      de fracassos anteriores ao r-ésimos sucesso.

    \item[Definição]: A v.a.\ $x$ tem distribuição Binomial negativa se, e somente se,
      sua f.m.p.\ é definida por:
      \begin{align}
        p(x)=P(X=x)=&
        \begin{cases}
          \binom{x+r-1}{r-1} p^r (1-p)^x, \text{\quad x=0,1,2,\dots} \\
          0, \text{\quad caso contrário}
        \end{cases}
      \end{align}
    \begin{center}Notação: $X \mathtt{\sim}BN(r,p)$\end{center}
      Temos que: 
      \begin{align}
        \mu_{x}=E(x)=\frac{r(1-p)}{p} \\
        \sigma^2 = var(x)=\frac{1-p}{p^2}
      \end{align}
    \item[Observações]: 
      \begin{enumerate}
        \item Note que se $r=1$, temos o modelo geométrico.
        \item $$\binom{x+r-1}{r-1} = \binom{x+r-1}{x}$$
      \end{enumerate}
    \item A Binomial Negativa pode ser definida como o $n^o$ de ensaios necessários para 
      a obtenção do r-ésimo sucesso. Formando $y=x+r$ temos a quantidade desejada e 
      seus valores variam de $r$ em diante. Assim sendo $Y$ o $n^o$ de ensaios até 
      a obtenção de $r$ sucessos, temos: 
      \begin{align}
        p(y)=p(Y=y)=&\begin{cases}
          \binom{y-1}{m-1}p^m(1-p)^{y-m}, \text{\quad }y=m,m+1,\dots \\
          0, \text{\quad caso contrário}
        \end{cases}
      \end{align}
    \item[Exemplo]: Em uma serie do campeonato de basquete, o time que ganhar quatro em sete jogos sera
      o vencedor. Suponha que o time $A$ tenha probabilidade de $55\%$ de ganhar de $B$ e que $A$ e $B$ 
      se enfrentarao em uma serie de sete jogos. Qual a probabilidade de que $A$ venca a serie em 6 jogos?
     \begin{align*} 
       X:\text{O número de derrotas de $A$ até que $A$ ganhe 4 partidas}\\
       X \mathtt{\sim}BN (r=4, p=0,55)
     \end{align*}
     \begin{align*}
       P(X=2)=\binom{2+4-1}{4-1} 0,55^{4}\times 0,45^{2}=0,1853
     \end{align*}
     Ou, de maniera similar, temos:
     \begin{align*}
       Y: \text{ número de partidas até que $A$ venca o campeonato}
     \end{align*}
     \begin{align*}
       P(Y=6)= \binom{6-1}{4-1} 0,55^{4} \times 0,45^{6-4} = 0,1853
     \end{align*}
 \end{description}
 \subsection{Distribuição Hipergeometrica}
\begin{description}
  \item
Considere um conjunto de $n$ objetos dos quais $m$ são do tipo $I$
e $(n-m) $ são do tipo $II$. Para um sorteio de $r$ objetos $(r<n)$, 
feitos ao acaso e sem repeticao, defina $X$ como o número de objetos do tipo
$I$ selecionados.
  \item[Definição:] A v.a.\ $X$ tem distribuição Hipergeometrica se sua f.m.p.\ é dada
    por:
    \begin{align*}
      P_x(x)= P(X=x)= \frac{\binom{m}{xs} \binom{n-m}{r-x}}{\binom{n}{r}}
    \end{align*}
    Em que $x$ (inteiro) é tal que:
    \begin{align*}
      max \{0,r-(n-m)\} \leq x \leq min \{ r,m \}
    \end{align*}
  \item[Observação]: Os limites de $x$ garantem que situaçoes absurdas ocorram.
  \item Temos que: 
    \begin{align*}
      E(X) = r \frac{m}{n} \\
      Var(x)= \frac{r\times m (n-m)(n-r)}{n^2 (n-1)}
    \end{align*}
    \begin{figure}[H]
      \centering
      \begin{tikzpicture}[>=latex,shorten >=2pt,shorten <=2pt,shape aspect=1]
\path[->]  (0,1.4) edge[bend left=40]  (1.6,0.8);
\node (m) [cylinder, shape border rotate=90, draw,minimum height=3cm,minimum width=2cm]
{$m$};
\node(n) [ below=0.3cm] {$(n-m)$};
\node(tipo1) [right= 1.4cm] {Tipo $\rom{1}$};
\node(tipo2) [below right= 0.3cm and 1.4cm] {Tipo $\rom{2}$};
\node(obj) [above right= 0.3 cm and 1.4cm] {$r$ objetos (sem reposição)};
\node(nobj) [below = 1.3 cm ] {$n$ objetos };
\draw[->] (0.2,0) -- (1.6,0);
\draw[->] (0.6,-0.6) -- (1.6,-0.6);


\end{tikzpicture}

      \caption{}
      \label{fig:x}
    \end{figure}
  \item[Notacao]: $X \mathtt{\sim} Hgeo (m,n,r)$
  \item[Exemplo]: Considere que em um lote de $20$ peças, existam $4$ defeituosas.
    Seleciona-se $5$ dessas peças, sem resposicao, qual seria a probabilidade 
    de duas defeituosas terem sido escolhidas?
\begin{figure}[H]
  \centering
  \begin{tikzpicture}[>=latex,shorten >=2pt,shorten <=2pt,shape aspect=1]
\path[->]  (0,1.4) edge[bend left=40]  (1.6,0.8);
\node (m) [cylinder, shape border rotate=90, draw,minimum height=3cm,minimum width=2cm]
{$4$};
\node(n) [ below=0.3cm] {$16$};
\node(tipo1) [right= 1.4cm] {Defeituosas};
\node(tipo2) [below right= 0.3cm and 1.4cm] {Não defeituosas};
\node(obj) [above right= 0.3 cm and 1.4cm] {$r=5$};
\node(nobj) [below = 1.3 cm ] {$20$ peças};
\draw[->] (0.2,0) -- (1.6,0);
\draw[->] (0.6,-0.6) -- (1.6,-0.6);


\end{tikzpicture}

  \label{fig:x2}
\end{figure}
    \begin{align*}
      X: \text{$N^{o}$  de peças defeituosas em $5$ retiradas }\\
      P(x=2)=\frac{\binom{4}{2} \binom{16}{3}}{\binom{20}{5}}\\
      R_{x}: \{ 0,1,2,3,4 \}
    \end{align*}
\end{description}
\subsection{Distribuição de Poisson}
\begin{description}
  \item É largamente empregada quando se deseja contar o número de eventos de certo tipo
que ocorrem em um certo período de tempo ou superficie ou volume.
\item[Definicao] Uma v.a.\ tem Distribuição Poisson com parâmetro $\lambda$, $\lambda >0$,se sua
  f.m.p.\ é dada por:
 \begin{align}
   P_{x} (x)= P(X=x)= \frac{e^{-\lambda}\lambda^{x}}{x!}, \quad x=0,1,2,\ldots
 \end{align} 
 \item{Notacao}: $X \mathtt{\sim} Poi(\lambda)$
 \item{Temos que}:
   \begin{align}
     E(X)= \lambda \\
     Var(x)= \lambda
   \end{align}
   \item[Exemplo]: Uma central telefonica recebe, em média, 5 chamadas por minuto.
     Supondo que a distribuição de Poisson seja adequada nessa situação, obtenha a 
     probabilidade de que a central receba no maximo 2 chamadas durante um intervalo
     de um minuto.
     \begin{align*}
       X: n^{o} \text{de chamadas recebidas em 1 minuto em uma central telefonica}
       \\
       E(X)=\lambda= \text{5 chamadas/minuto}\\
     \end{align*}
     \begin{align*}
       P(X \leq 2)= P(X=0)+ P(X=1)+P(X=2)
     \end{align*}
     \begin{align*}
       P(X=0) = \frac{e^{-5}5^0}{0!}= 0,0067\\
       P(X=1) = \frac{e^{-5}5^1}{1!}= 0,0334\\
       P(X=2) = \frac{e^{-5}5^2}{2!}= 0,0842
     \end{align*}
     Logo,
     \begin{align*}
       P(X \leq 2)= 0,0067+0,0334+0,0842=0,1243
     \end{align*}
   \end{description}
     \subsubsection{O processo de Poisson}
     \begin{description}  
       \item     Suponha que $\mu$ seja a média de ocorrência do evento de interesse
     em $t$ unidades de medida(por exemplo, tempo). Denotamos por $\lambda$,
     a taxa média de ocorrência em uma unidade de medida, como $\mu= \lambda t$.
     Podemos reescrever a f.m.p.\ por:
     \begin{align*}
       p_{x} =P(X=x)= \frac{e^{-\lambda t} \left( \lambda t \right)^{x}}{x!}, \quad x=0,1,2,\ldots
     \end{align*}
       \item[Exemplo]: Considere o exemplo anterior e calcule a probabilidade de que a central 
         telefonica receba no maximo duas chamadas em 4 minutos.
         \begin{align*}
           P(X \leq 2 \/ t=4) = P(X=0)+P(X=1)+P(X=2)
         \end{align*}
     \begin{align*}
       P(X=0) = \frac{e^{-5 \times 4}(5\times 4)^0}{0!}= 2,06 \times 10^{-9}\\
       P(X=1) = \frac{e^{-5 \times 4}(5\times 4)^1}{1!}= 4,12 \times 10^{-8}\\
       P(X=2) = \frac{e^{-5 \times 4}(5\times 4)^2}{2!}= 4,12 \times 10^{-7} 
     \end{align*}
    \begin{align*}
      P(X \leq 2 \/ t=4)= 2,06 \times 10^{-9}+4,12 \times 10^{-8}+4,12 \times 10^{-7} = 4,56 \times 10^{-7} \approx 0
    \end{align*} 
  \item[Resultados Importantes]
    \begin{description}
  \item [Resultado 1]: Se $X_1, X_2,\ldots, X_n$ são independetes e $X_i \mathtt{\sim} Poisson(\lambda_i)$, então:
    \begin{align}
      Y= X_1+X_2+\ldots+X_n \mathtt{\sim} Poisson\left( \lambda_1+\ldots+\lambda_n \right)
    \end{align}
  \item [Resultado 2]: Se $X \mathtt{\sim}Bin(M,P)$. com $m$ grande e $P$ pequeno, pode-se aproximar a distribuição de 
    $X$ pela distribuição de Poisson, cujo parâmetro sera $\lambda = M \times P$.
\end{description}

      \item [Exemplo]: número de gols marcados em $M=100$ tentativas.

    \begin{align*}
      P=0,05 \quad \therefore \quad  X \mathtt{\sim} Bin(100, 0,05) \\
      P(X \ge 50)=P(X=50), \ldots + P(X=100)= 1- \{P(X=0)+\ldots+P(X=49) \}\\
      E(X)=M \times p = \mu\\
      X \mathtt{\sim}Poi(\lambda= M \times P)
    \end{align*}

  \item[Exercício]: Em certa instalacao industrial, acidentes ocorrem com baixa frequência. Sabe-se que 
    a probabilidade de um acidente ocorrer em um certo dia é $0,005$ e que os acidentes são independentes.
    
    \begin{enumerate}[label=(\alph*)]
      \item Qual a probabilidade de que em um período de 400 dias haja no maximo 3 dias com acidente?
      Utilize a aproximacao pela distribuição de Poisson.
    \item Calcule a probabilidade exata do item anterior e compare os resultados.
   \end{enumerate}
   \end{description}
   \section{Modelos Contínuos}
   \subsection{Distribuição Uniforme} 
   \begin{description}
     \item [Definição]: Uma v.a.\ contínua $X$ tem distribuição uniforme no intervalo $(a,b), \quad a,b \in \mathbb{R}$,
       se sua f.d.p.\ é dada por:
\begin{align}
  f_x(x) = \begin{cases}
\frac{1}{b-a}, \quad a \leq x \leq b  \\
   0, \quad \text{caso contrário}
  \end{cases}
\end{align}
\item[Notação]: $X \mathtt{\sim} U(a,b)$
  \begin{figure}[H]
    \centering
    \pgfmathdeclarefunction{weird}{1}{%
  \pgfmathparse{5}%
}

\begin{tikzpicture}

\def\startx{-7}
\def\endx{7}
\def\verticalbar{6}
\begin{axis}[no markers,
axis lines*=left, xlabel=Variavel aleatoria Continua, ylabel=f.d.p,,
height=6cm, width=10cm,  xticklabels={a,\empty,\empty,$a_1$,\empty,\empty,\empty,\empty,\empty,$b_1$,\empty,\empty,\empty,\empty,b},
xtick={-7,-6,-5,-4,-3, -2, -1, 0, 1, 2, 3,4,5,6,7}, ytick=\empty,xtick style={draw=none},
enlargelimits=false, clip=false, axis on top,
    domain=\startx:\endx
]
\addplot [name path=g5,thin, smooth,] {weird(1)};
\addplot [
    fill=orange!52,
    draw=none,
    domain=-4:2,
    stack plots=y
    ] {max(weird(1),0)} \closedcycle;
\node[below] at (590,-40) {$P(a_1<x<b_1)$};
\draw[dashed,color=blue!60] (825,-100) to (825,0);
\node[above,color=blue!60] at (825,0) {$P(X=k)=0$};
\end{axis}
\end{tikzpicture}

    \caption{ilustração da f.d.p de $X$}
  \end{figure}
\item A f.d.a.\ é dada por:
  \begin{align}
    F_x{x} = \begin{cases}
      0, \quad \text{se} x<a \\
      \frac{x-a}{b-a}, \quad \text{se} a\leq<b\\
      1, x \ge b
    \end{cases}
  \end{align}
  \begin{align}
    P(X \leq x) = \int \limits_{(- \infty}^{x} f_{x} (t) dt = \int_{a)}^{(x} \frac{1}{b-a)} dt\\
    \nonumber = \frac{1}{b-a}t |_{a}^{x} = \frac{x^a -a}{b-a}
  \end{align}
\item[Exemplo]:
  \begin{align*}
    P(a_1 \leq X \leq b_1) = F_{x} (b_1) - F_{x}(a_1) \\
    P(X > a_1)= 1- P(X \leq a_1) \\
    = 1-F_{x} (a_1) \\
    P(X > a_1)= F_{x}(a_1)
  \end{align*}
 \item A média de $x$ é:
   \begin{align*}
     E(X) = \frac{a+b}{2}
   \end{align*}
 \item E a variância de $x$ é:
   \begin{align*}
     Var(x) = \frac{ \left(b-a\right)^2 }{12}
   \end{align*}
   \item[Exemplo]: Seja $X$ uma v.a.\ com distribuição uniforme, $U(-1/2 , 1/2). $
     Calcule:
\begin{enumerate}[label=(\alph*)]
  \item $F_{x} (x)$
  \item $P(\frac{-1}{4} \leq x \leq \frac{1}{4})$
  \item $E(x)$ e $Var(x)$
\end{enumerate}
Sabe-se que:
\begin{align*}
  X \mathtt{\sim}  U(\frac{-1}{2}; \frac{1}{2})
\end{align*}
Assim, a f.d.p.\ de $x$ é:
\begin{align*}
  f_{x} (x)= \frac{1}{\frac{1}{2}- \left( \frac{-1}{2}\right)} = 1 \quad \frac{-1}{2}<x<\frac{1}{2}
\end{align*}
\begin{enumerate}[label=(\alph*)]
  \item  $F_{x}(x)=?$
    \begin{align*}
      F_{x} (x)= P(X \leq x) = \int \limits_{( -\infty  )}^{(x)} f(t)dt\\
      = \int \limits_{(\frac{-1}{2})}^{(x} 1 dt= t|_{\frac{-1}{2}}{x)} \\
      = x+ \frac{1}{2}
    \end{align*}
    \begin{align*}
      F_{x} (x) = P(X \leq x) = \begin{cases}
        0, \quad \text{se}  x< \frac{-1}{2} \\
        x+ \frac{1}{2} , \quad \text{se} \frac{-1}{2} \leq x < \frac{1}{2} \\
        1, \quad x \ge \frac{1}{2}
      \end{cases}
    \end{align*}
    \item 
      \begin{align*}
      P(\frac{-1}{4}\leq X \leq \frac{1}{4})  = \int \limits_{(\frac{-1}{4})}^{(\frac{1}{4})}\\
      = x|_{\frac{-1}{4}}^{\frac{1}{4}}= \frac{1}{4}+\frac{1}{4}=\frac{1}{2}
    \end{align*}
    Ou 
    \begin{align*}
      P(\frac{-1}{4} \leq X \leq \frac{1}{4})= F_{x}(\frac{1}{4})-F_{x}(\frac{-1}{4}) \\
      \frac{3}{4}- \frac{1}{4}= \frac{1}{2}
    \end{align*}
    \item 
      \begin{align*}
        E(X) = \frac{a+b}{2} \\
        = \frac{\frac{-1}{2} + \frac{1}{2}}{2}=0\\
        Var(x)= \frac{ \left( b-a \right)^2 }{12}= \frac{ \left( \frac{1}{2} + \frac{1}{2} \right)^2 }{12}= \frac{1}{12}
      \end{align*}
\end{enumerate}

   \end{description}
\subsection{Distribuição Exponencial}
\begin{description}
  \item [Definição]: Uma v.a.\ contínua $X$ tem distribuição exponencial com parâmetro $\lambda, \lambda>0$, se sua
    f.d.p.\ é dada por:
    \begin{align*}
      f_{x} (x)= \begin{cases}
        \lambda e^{- \lambda x} , \quad x \ge 0 \\
        0, \quad \text{caso contrário}
      \end{cases}
    \end{align*}
  \item[Notação]: $X \mathtt{\sim} Exp(\lambda)$
    \begin{figure}[H]
      \centering
      \pgfmathdeclarefunction{dnorm}{2}{%
  \pgfmathparse{exp(-((x-#1)/7)}%
}
\pgfmathdeclarefunction{weird}{1}{%
  \pgfmathparse{-1(-3*x^4- 35*x^2+3*x+12+100*cos(300*x )+139*sin(238*x)-120*cos(149*x)-194*sin(489*x)+9000) }%
}

\begin{tikzpicture}

\def\startx{-7}
\def\endx{7}
\def\verticalbar{6}
\begin{axis}[no markers, samples=100,
  axis lines*=left, xlabel=x, ylabel=$f_{x}(x)$,
height=6cm, width=10cm,  xticklabels={\empty,\empty,\empty,\empty,a,\empty,\empty,\empty,\empty,b,\empty,\empty,\empty,\empty,\empty},
xtick={-7,-6,-5,-4,-3, -2, -1, 0, 1, 2, 3,4,5,6,7}, xtick style={draw=none},ytick =\empty,
 clip=false, axis on top,
    domain=\startx:\endx
]
\addplot [name path=g5,thin, smooth] {dnorm(1,2)};
\addplot [
    fill=green!52,
    draw=none,
    domain=-3:2,
    stack plots=y
    ] {max(dnorm(1,2),0)} \closedcycle;
\node[above] at (650,3) {$P(a<x<b)$};
\end{axis}
\end{tikzpicture}

    \end{figure}
    \item A f.d.a.\ de $x$ é:
      \begin{align*}
        P(X \le x)  = \int \limits_{ -\infty }^{x} f_{x}(t)dt \\
        = \int \limits_{0}^{x} \lambda e^{-\lambda t } dt \\
        = \eval{-e^{-\lambda t}}{0}{x} = -e^{-\lambda x} +1\\
        = 1- e^{-\lambda x}
      \end{align*}
      \begin{align*}
        F_{x} (x) = \begin{cases}
         0, \quad x<0\\
         1-e^{-\lambda x} , x\ge 0
        \end{cases}
      \end{align*}
    \item A média e variância de x são:
      \begin{align*}
        E(x) = \frac{1}{\lambda}
      \end{align*}
      \begin{align*}
        Var(x) = \frac{1}{\lambda^2}
      \end{align*}
    \item[Propriedade]:
       Se $X \mathtt{\sim}  Exp(\lambda)$, então:
        \begin{align*}
          P(X> a+b  \/ x>b) = P(X > a)
        \end{align*}
        Esta propriedade é conhecida por falta de memoria e é a única distribuição 
        contínua que tem essa propriedade.

        \begin{description}
      \item        [Observação]:

        Outra parametrização para a distribuição exponencial é:

        \begin{align*}
          f_{x} (x)= \frac{1}{2}e^{-\frac{x}{2}}, \quad x\ge 0
        \end{align*}
        Ou seja, $\lambda= \frac{1}{\alpha}$
        A média de x sera:
        \begin{align*}
          E(x)  = \alpha
        \end{align*}
        E a variância é dada por:
        \begin{align*}
          Var(x) = \alpha^2
        \end{align*}
      \item [Exemplo:] Seja $X$ o tempo de vida util de um fusivel que tem distribuição Exponencial
        com vida média de 100 horas. Qual a probabilidade de um fusivel durar mais de 150
        horas?
        \begin{align*}
          X \sim  Exp(\frac{1}{100}) \\
          \mu = E(x)= \frac{1}{\lambda}=100 \left( \lambda= \frac{1}{100} \right)\\
          P(X> 150)= \int \limits_{150}^{\infty} \frac{1}{100} e^{- \frac{1}{100} x} dx =0,22313
        \end{align*}
        Ou 
        \begin{align*}
          P(X>150) = 1- P(X \le 150)=\\ 1- \int \limits_{0}^{150} \frac{1}{100} e^{- \frac{1}{100}x}dx          = 0,22313
        \end{align*}
      \end{description}
\end{description}
\subsection{Distribuição Normal}
E a distribuição mais importante dos modelos probabilisticos. 
é tambem conhecida como distribuição gaussiana. Sua representação 
gráfica é conhecida por curva em forma de sino.
\begin{description}
  \item [Definição:]  Uma v.a.\ contínua $X$ tem Distribuição normal com parâmetro $\mu$ (média) e 
    $\sigma^2$ se sua f.d.p.\ é dada por:
    \begin{align}
      f_{x} (X)= \frac{1}{\sqrt{2\pi \sigma^2}} e^{\frac{-1}{2\sigma^2} \left( x-\mu \right)^2} \quad -\infty<x<\infty \\
      \text{Com } \quad -\infty < \mu < \infty, \quad \sigma^2>0
    \end{align}
  \item [Notação:] $X \sim N(\mu ; \sigma^2)$

Uma ilustração gráfica da sua f.d.p.\ é:
\begin{figure}[H]
  \centering
  \pgfmathdeclarefunction{gauss}{2}{\pgfmathparse{1/(#2*sqrt(2*pi))*exp(-((x-#1)^2)/(2*#2^2))}%
}
\begin{tikzpicture}
\begin{axis}[no markers, domain=0:10, samples=100,
  axis lines*=left, xlabel=$x$, ylabel=$f_{x}(x)$,,
height=6cm, width=10cm,xticklabels={\empty,\empty,$a$,$\mu$,$b$,\empty,\empty},
xtick={-3, -2, -1, 0, 1, 2, 3}, ytick=\empty,
enlargelimits=false, clip=false, axis on top
]

\addplot [color=blue, domain=-3:3] {gauss(0,1)} ;
\addplot [fill=cyan!20, draw=none, domain=-1:1] {gauss(0,1)} \closedcycle;
\addplot[color=red] coordinates {(-3,0.2) (0,0.2)};
\addplot[color=red] coordinates {(0,0.4) (3,0.4)};
\node[coordinate, pin={0.5}] at (axis cs: -1.6, 0.2){};
\node[coordinate, pin={0.5}] at (axis cs: 1.5, 0.4){};
\end{axis}
\end{tikzpicture}

\end{figure}
A média e variância de $x$ são:
\begin{align*}
E(x) =\mu \\
Var(x)=\sigma^2
\end{align*}
\item [Propriedades:]
A distribuição é simetrica em relação à média. Isto é:
\begin{align}
  f(\mu - x) = f(\mu+x), \quad \forall x \in \mathbb{R}
\end{align}
\item Como a área total sob a curva é igual a 1, à esquerda e à direita de $\mu$, 
  a área é igual a $0,5$.
\item 
  \begin{align*}
    P(\mu -\sigma \leq X \leq \mu+\sigma)= 0,6896\\
    P(\mu -2\sigma \leq X \leq \mu+2\sigma)= 0,9546\\
    P(\mu -3\sigma \leq X \leq \mu+3\sigma)= 0,9973
  \end{align*}
Ilustrando:
%Grafico aqui
%http://johncanning.net/wp/?p=1202
\begin{figure}[H]
  \centering
  \pgfmathdeclarefunction{gauss}{2}{\pgfmathparse{1/(#2*sqrt(2*pi))*exp(-((x-#1)^2)/(2*#2^2))}%
}
\begin{tikzpicture}
\begin{axis}[no markers, domain=0:10, samples=100,
axis lines*=left, xlabel=Desvio Padrão, ylabel=Frequência,,
height=6cm, width=10cm,
xtick={-3, -2, -1, 0, 1, 2, 3},xticklabels={$\mu-3\sigma$,$\mu-2\sigma$,$\mu-\sigma$,$\mu$,$\mu+\sigma$,$\mu+2\sigma$,$\mu+3\sigma$}, ytick=\empty,
enlargelimits=false, clip=false, axis on top,
grid = major ]
\addplot [fill=cyan!20, draw=none, domain=-3:3] {gauss(0,1)} \closedcycle;
\addplot [fill=orange!20, draw=none, domain=-3:-2] {gauss(0,1)} \closedcycle;
\addplot [fill=orange!20, draw=none, domain=2:3] {gauss(0,1)} \closedcycle;
\addplot [fill=blue!20, draw=none, domain=-2:-1] {gauss(0,1)} \closedcycle;
\addplot [fill=blue!20, draw=none, domain=1:2] {gauss(0,1)} \closedcycle;
\addplot[] coordinates {(-1,0.4) (1,0.4)};
\addplot[] coordinates {(-2,0.3) (2,0.3)};
\addplot[] coordinates {(-3,0.2) (3,0.2)};
  \node[coordinate, pin={68.2\%}] at (axis cs: 0, 0.4){};
\node[coordinate, pin={95\%}] at (axis cs: 0, 0.3){};
\node[coordinate, pin={99.7\%}] at (axis cs: 0, 0.2){};
\node[coordinate, pin={34.1\%}] at (axis cs: -0.5, 0){};
\node[coordinate, pin={34.1\%}] at (axis cs: 0.5, 0){};
\node[coordinate, pin={13.6\%}] at (axis cs: 1.5, 0){};
\node[coordinate, pin={13.6\%}] at (axis cs: -1.5, 0){};
\node[coordinate, pin={2.1\%}] at (axis cs: 2.5, 0){};
\node[coordinate, pin={2.1\%}] at (axis cs: -2.5, 0){};
\end{axis}
\end{tikzpicture}

\end{figure}
A f.d.a.\ de uma v.a.\ $X \sim N(\mu,\sigma^2)$ é:
\begin{align*}
  F_{x} (X)= P(X \leq x)= \int^{x}_{-\infty} \frac{1}{\sqrt{2\pi \sigma^2}}e^{\frac{-1}{2\sigma^2} \left( t- \mu \right)^2} dt 
\end{align*}
Cuja integral nao tem solução analítica. Assim, calculamos suas probabilidades com auxilio de tabelas.

\item [Definição]: Se $X \sim N(\mu; \sigma^2)$, entao a v.a.\ $Z$ é definida por:
  \begin{align}
   Z= \frac{X-\mu}{\sigma} 
  \end{align}
  Terá a distribuição normal com média 0 e variância 1. Ou seja, $Z \sim N(0,1)$.
  
  Essa distribuição é conhecida como distribuição normal-padrão ou reduzida. 
\end{description}
  \subsubsection{Cálculo de probabilidades}
  \begin{description}
    \item [Exemplo] Uso da tabela normal padrão:

      Seja $Z\sim N(0,1)$ calcule:
      \begin{enumerate}[label=(\alph*)]
        \item $P(0 \leq Z \leq 1,65)$
        \item $P(Z \leq 0,5)$
        \item $P(Z < -1,57)$
        \item $P(-0,65 \leq Z \leq 0,5)$
        \item $P(0,8 \leq Z < 1,4)$
        \item $P(0 \leq Z \leq z)=0,4753$
        \item $P(Z < z)=0,05$
      \end{enumerate}
      
      Resolva:
\begin{enumerate}[label=(\alph*)]
        \item $P(0 \leq Z \leq 1,65)$
          \begin{figure}[H]
            \centering
            \input{tikz/ModelosProb/dnorm3.tikz}
          \end{figure}
          %Olhando a primeira tabela:
        \begin{align*}          P(0 \leq Z \leq 1,65)= 0,45053 \end{align*}
          %Olhando a segunda tabela:
          Ou:
          \begin{align*}
          P(0 \leq Z \leq 1,65)= P(Z < 1,65)-P(Z \leq 0) \\
          = P(z < 1,65)- 0,5 \\
          = 0,9505 - 0,5 = 0,4505
        \end{align*}
      \item $P(Z < 0,5)$

\begin{figure}[H]
            \centering
            \input{tikz/ModelosProb/dnorm4.tikz}
          \end{figure}

%grafico aqui
        \begin{align*}
          P(Z < 0,5)= 0,5 + P(0 \leq Z \leq 0,5) \\
          = 0,5 + 0,19146 \\
          = 0,69146
      \end{align*}
      Ou 
      \begin{align*}
        P(Z <0,5) = 0,6915
      \end{align*}

    \item $ P(Z< -1,57)$
\begin{figure}[H]
            \centering
            \input{tikz/ModelosProb/dnorm5.tikz}
          \end{figure}

Ou equivalentemente:
\begin{figure}[H]
  \centering
  \input{tikz/ModelosProb/dnorm51.tikz}
\end{figure}
      %grafico aqui
      \begin{align*}
        P(Z< -1,57) = P(Z> 1,57)\\
        =0,5 - P(0 \leq Z \leq 1,57) \\
        = 0,5 - 0,44179 \\
        = 0,0582
      \end{align*}
      Ou 
      \begin{align*}
        P(Z < -1,57) = 0,0582
      \end{align*}
    \item $P(-0,65 \leq Z \leq 0,65)$
\begin{figure}[H]
            \centering
            \input{tikz/ModelosProb/dnorm6.tikz}
          \end{figure}

      %Grafico aqui
      \begin{align*}
        P(-0,65 \leq Z \leq 0) = P(-0,65 \leq Z \leq 0) \\
        =2 P(0 \leq Z \leq 0,65) \\
        =2 \times 0,24215 \\
        =0,4843
      \end{align*}
      \begin{align*}
        P( -0,65 \leq Z \leq -0,65 ) = P(Z < 0,65) - P(Z < -0,65)
      \end{align*}
    \item 
\begin{figure}[H]
            \centering
            \input{tikz/ModelosProb/dnorm7.tikz}
          \end{figure}

      %grafico aqui
      \begin{align*}
        P(0,8 < z < 1,4)  = P(0 \leq Z \leq 1,4) - P(0 \leq Z \leq 0,8) \\
        = 0,41924 - 0,28814
      \end{align*}
    \item $P(0 \leq Z \leq z) = 0,4753$
\begin{figure}[H]
            \centering
            \input{tikz/ModelosProb/dnorm8.tikz}
          \end{figure}

      %grafico aqui
      \begin{align*}
       z= 1,965 
      \end{align*}
    \item $P(Z < z) = 0,05$ \begin{figure}[H] \centering
            \pgfmathdeclarefunction{gauss}{2}{\pgfmathparse{1/(#2*sqrt(2*pi))*exp(-((x-#1)^2)/(2*#2^2))}%
}
\def\startx{-3}
\def\endx{-1.645}
\begin{tikzpicture}
\begin{axis}[no markers, domain=0:10, samples=100,
axis lines*=left, xlabel=$x$, ylabel=$f_{x}(x)$,
height=6cm, width=10cm,xtick={\endx}, ytick=\empty,
enlargelimits=false, clip=false, axis on top
]

\addplot [color=blue, domain=-3:3] {gauss(0,1)} ;
\addplot [fill=cyan!20, draw=none, domain=\startx:\endx] {gauss(0,1)} \closedcycle;
\end{axis}
\end{tikzpicture}

          \end{figure}

      %Grafico aqui
      \begin{align*}
       z= -1,645 
      \end{align*}
      \end{enumerate}

    \item [Exemplo:] Seja $X \sim N(90,100)$ \\
      \begin{description}
        \item  [Determine]:
      \begin{enumerate}[label=(\alph*)]
        \item $P(80 < x < 100)$
        \item $P(x \geq  90)$
        \item $P(60 \leq x \leq 75 )$
      \end{enumerate}

    \item [Resolução]:
      \begin{enumerate}[label=(\alph*)]
        \item $P(80 < x < 100)$
          \begin{figure}[H]
            \centering
            \input{tikz/ModelosProb/dnorm10.tikz}
            \label{fig:}
          \end{figure}
          \begin{align*}
            Z= \frac{X-\mu  }{\sigma} \\
            z_1 = \frac{80-90}{10}=-1\\
            z_2 = \frac{100-90}{10}=1
          \end{align*}
                  \begin{align*}
           P(80<X < 100) = P(-1 < Z <1)
         \end{align*} 
 \begin{figure}[H]
            \centering
            \input{tikz/ModelosProb/dnorm11.tikz}
            \label{fig:}
          \end{figure}

         \begin{align*}
           P(-1 < Z < 1)  = 2 \times P(0<Z < 1) \\
           = 2 \times 0,34134 \\
           =0,6826
         \end{align*}

        \item $P(x \geq  90)$
 \begin{figure}[H]
            \centering
            \input{tikz/ModelosProb/dnorm12.tikz}
            \label{fig:}
          \end{figure}
          \begin{align*}
            P(x \geq 90) =P(Z \geq \frac{90-90}{10}) \\
            = P(Z \geq 0)=0,5
          \end{align*}

        \item $P(60 \leq x \leq 75 )$
 \begin{figure}[H]
\centering
\input{tikz/ModelosProb/dnorm13.tikz}
\label{fig:}
\end{figure}
 \begin{figure}[H]
            \centering
            \input{tikz/ModelosProb/dnorm14.tikz}
            \label{fig:}
          \end{figure}


          \begin{align*}
            P(60 \leq X \leq 75 ) \\
            =P(\frac{60-90}{10} \leq Z \leq \frac{75-90}{10}) \\
            =P(-3 \leq Z \leq 1,5)
          \end{align*}
          %Grafico aqui
          \begin{align*}
            P(-3 < Z < -1,5)\\
            =P(Z <1,5)- P(Z< -3) \\
            =0,0668- 0,0013 \\
            =0,0655
          \end{align*}

      \end{enumerate}
    \item [Exemplo:] Seja $X \sim \mathtt{N}(50; 10^2)$
      \begin{description}
        \item [Determine:]
          \begin{enumerate}[label=(\alph*)]
   \item $P(|X-50| <10)$
  \item $P( \mu-a \leq X \leq \mu +a )=0,9$
  \item $P(K< X <70)=0,8185$
  \item $P(K < x <75)= 0,3031$
          \end{enumerate}
        \item [Resolução:]
\begin{enumerate}[label=(\alph*)]
   \item $P(|X-50| <10)$
     \begin{align*}
       \begin{cases}
        X-50<10 \Rightarrow X<60 \\
        -(X-50) <10 \Rightarrow X>40
       \end{cases}
     \end{align*}
     \begin{align*}
       P(|X-50| <10)  = P(40<X<60) \\
   = P(\frac{40-50}{10}< Z < \frac{60-50}{10}) \\
   = P(-1 < Z < 1)
 \end{align*} 
 \begin{figure}[H]
            \centering
            \input{tikz/ModelosProb/dnorm15.tikz}
            \label{fig:}
          \end{figure}
 \begin{figure}[H]
            \centering
            \input{tikz/ModelosProb/dnorm16.tikz}
            \label{fig:}
          \end{figure}

 %grafico aqui
\begin{align*}
  P(-1 < Z <1)  = 2 \times P(0 < Z <1) \\
  = 2 \times 0,34124 \\
  \simeq 0,68
\end{align*}

  \item $P\left( \mu-a \leq X \leq \mu +a \right)=0,9$
\begin{align*}
  P\left(50- a \leq X \leq 50 +a\right)  = 0,9 \\
  P\left(\frac{(50-a)-50}{10}\right) \leq Z \leq \frac{(50+a)-50}{10} \\
  P\left(\frac{-a}{10} \leq Z \leq \frac{a}{10}\right)=0,9 
\end{align*}

 \begin{figure}[H]
            \centering
            \input{tikz/ModelosProb/dnorm17.tikz}
            \label{fig:}
          \end{figure}

%grafico aqui
\begin{align*}
  P( \frac{-a}{10} \leq Z \leq \frac{a}{10} ) = 2 \times P(0 \leq Z \leq \frac{9}{10})= 0,9 \\
  P(0 \leq Z \leq \frac{a}{10})=0,45 \\
  \frac{a}{10}= 1,645 \\
  a = 16,5
\end{align*}
  \item $P(K< X <70)=0,8185$
 \begin{figure}[H]
            \centering
            \input{tikz/ModelosProb/dnorm18.tikz}
            \label{fig:}
          \end{figure}


    %grafico aqui
    \begin{align*}
      P( \frac{K-50}{10} < Z < \frac{70-50}{10} ) = 0,8185 \\
      P(\frac{K-50}{10} < Z < 2) =0,8185
    \end{align*}
    %grafico aqui
    \begin{align*}
      P( \frac{K-50}{10}<Z<2 )  = 0,8185 \\
      P( \frac{K-50}{10} < Z <0) + P(0 < Z <2) =0,8185 \\
      P(\frac{K-50}{10}<Z < 0) = 0,8185 - 0,47725 \\
      P( \frac{K-50}{10} < Z < 0 ) =0,3413
    \end{align*}
    %grafico aqui
    \begin{align*}
      \frac{-(K-50)}{10} = 10 \\
      K= 40
    \end{align*}
  \item $P(K < x <75)= 0,3031$
    %grafico aqui

    Temos o caso da tabela 1 caso $P(50 < X < 75) > 0,3031$.
    
    Do contrário, se $P(50 < X < 75) < 0,3031$, temos o caso do tabela 2.
    \begin{align*}
      P(50 < X < 75)  \\
      =P(\frac{50-50}{10} < Z < \frac{75-50}{10}) \\
      =P(0 < Z < 2,5) = 0,49379
    \end{align*}
    %Grafico aqui
    \begin{align*}
      P(K < X < 75)  = 0,3031\\
      =P(\frac{K-50}{10}< Z <2,5)= 0,3031
    \end{align*}
    %Grafico aqui
    \begin{align*}
      P(0 < Z < 2,5)  - P(0 < Z  < \frac{k-50}{k})=0,3031 \\
      0,49379 - 0,3031 = P(0 < Z < \frac{k-50}{10}) \\
      P(0<Z < \frac{\underbrace{k-50}_{0,495}}{10})= 0,19069  \\
      \frac{K-50}{10}= 0,495 \Rightarrow  K=4,95 +50 \\
      K=54,95
    \end{align*}
\end{enumerate}

    \end{description}
  \end{description}
  \subsubsection{Distribuição da Combinacao linear de v.a.\`s normais independentes}
  \begin{description}
    \item Sejam $X_1, X_2 , \ldots, X_n $ e  $n$ v.a.\`s tais que:
  \begin{align*}
    X_i \sim \mathtt{N}(\mu ; \sigma^2)  \quad \text{para} \quad i=1,2,\ldots,n
  \end{align*}
  Considere $Y= \sum_{i=1}^{n} X_i = X_1+\ldots+X_n$, temos que:
  \begin{align*}
    E(Y)  = E( \sum \limits_{(i)}^{n} X_i ) = \sum_{i=1}^{n} E(X_i) \\
    = \sum_{i=1}^{n} \mu = n \times \mu
  \end{align*}
  Ou
  \begin{align*}
    E(Y )  =E( \sum_{i=1}^n X_i )= E(X_1 + \ldots +X_n) \\
    = E(X_1) +E(X_2) + \ldots +E(X_n)\\
    =\underbrace{\mu + \mu + \dots + \mu}_{n} = n \times \mu
  \end{align*}
  Assim:
  \begin{align*}
    Var(Y) = Var(\sum_{i=1}^{n} X_i)= \sum^{n}_{i=1} Var(X_i) \\
    = \sum_{i=1}^{n} \sigma^2 = n  \times \sigma^2
  \end{align*}
  Logo, $Y \sim \mathtt{N}(n\mu ; n \sigma^2)$.

  Pelas propriedades da distribuição normal, temos que se $Y\sim \mathtt{N}(n\mu; n\sigma^2)$, 
  a variável reduzida sera:
  \begin{align*}
    Z= \frac{Y- E(Y)}{\sqrt{Var(Y)}} = \frac{Y- n\mu}{\sqrt{n\sigma^2}} \sim \mathtt{N}(0,1)
  \end{align*}
  Analogamente, temos a média amostral:
  \begin{align*}
    \bar{X} = \frac{\sum \limits_{(i)}^{n}}{X_i}
  \end{align*}
  Com isso, temos:
  \begin{align*}
    E(\bar{X}) = E( \frac{\sum_{i=1}^n X_i}{n} )= \frac{1}{n}\sum_{i=1}^{n} E(X_i) \\
    = \frac{1}{n} \sum_{i=1}^n \mu = \frac{1}{n} \times n \times \mu = \mu
  \end{align*}
  E também:
  \begin{align*}
    Var(\bar{X}) = Var(\frac{\sum_{i=1}^n}{n}) \\
    = \frac{1}{n^2} \sum_{i=1}^{n} Var(X_i) \\
    = \frac{1}{n^2} \sum_{i=1}^{n} \sigma^2 = \frac{1}{n^2} \times n \times \sigma^2 \\
    = \frac{\sigma^2}{n}
  \end{align*}
  Logo $\bar{X} \sim \mathtt{N} ( \mu ; \frac{\sigma^2}{n} ) $

  A variável reduzida é:
  \begin{align*}
    Z = \frac{\bar{X}- E(\bar{X})}{\sqrt{Var(\bar{X})}} = \frac{\bar{x} - \mu}{\sqrt{\frac{\sigma^2}{n}}} \\
    = \frac{\bar{x}- \mu}{\frac{\sigma}{\sqrt{n}}} \\
    Z= \frac{\sqrt{n} (\bar{X} -\mu)}{\sigma} \sim \mathtt{N} (0,1)
  \end{align*}
\item [Exemplo:] O peso de sacos de parafusos empacotados por uma maquina tem média de $50g$ e desvio-padrão de $2g$.
  Assumindo que  o peso tem distribuição normal, qual a probabilidade de que 10 sacos pesem juntos de $490g$? 
  Qual a probabilidade de que a média desses 10 sacos sejam no maximo $52g$?

\begin{enumerate}[label=(\alph*)]
  \item
  \begin{align*}
    X \sim \mathtt{N} (50, 2^2) \\
    X_1, X_2, \ldots , X_{10 } \\
    Y= \sum_{i=1}^{n} X_i < 490 \\
    Y \sim \mathtt{N}(500 ; 10 \times 2^2) \\
    P(Y <490 ) = P(Z < \frac{490-500}{\sqrt{40}}) \\
    P(Z< -1,58) = 0,0571
  \end{align*}
  %grafico aqui

\item $\bar{X \leq 52g}$
  \begin{align*}
    \bar{X}  = \frac{\sum X_{i}}{n} \sim \mathtt{N}(\mu ; \frac{\sigma^2}{n}) \\
    \bar{X} \sim N(50; \frac{2^2}{10})
  \end{align*}
  \begin{align*}
    P(X \leq 52)  = P(Z \leq \frac{52-50}{\sqrt{\frac{2}{5}}}) \\
    = P(Z \leq 3,16)
  \end{align*}
  %Grafico 
  \begin{align*}
    P(Z \leq 3,16 ) = 0,9992
  \end{align*}
\end{enumerate}
\end{description}
\end{description}
\section{Variaveis Aleatorias Bidimensionais}
\begin{description}
  \item Seja $X$ e $Y$ duas variáveis aleatórias quando ha interesse na variacao conjunta de $X$ e $Y$, estudamos o par $(X,Y)$ como uma variável aleatória bidimensional. `1`
  \item  [Definição:] A variável aleatória bidimensional discreta $(X,Y)$ tem funcao massa de probabilidade conjunta definida por:
    \begin{align}
      P_{x,y}= P(X=x, Y=y)
    \end{align}
    A funcao de distribuição acumulada $F_{x,y} = P(X \leq x, Y \leq y)$ para $(x,y)$ valores que $(X,Y)$ pode assumir.
  \item [Propriedades]:
    \begin{enumerate}[label=(\alph*)]
      \item $p_{x,y}(x,y) \geq 0, \quad \forall (x,y)$.
      \item $\sum \limits_{R_X} \sum \limits_{R_Y} p_{x,y} (x,y) = 1$.
    \end{enumerate}
  \item [Exemplo]:
Em um processo de inspecao de qualdiade, cada peca passa por dois testes e eh classificada de acordo com duas variáveis $X$ e $Y$, tais que:
\begin{align*}
 X= \begin{cases}
   1,\quad \text{Se a peca passa no primeiro teste}  \\
   0, \quad \text{Se a peca falha no primeiro teste}
 \end{cases} 
\end{align*}
\begin{align*}
 Y= \begin{cases}
  1,\quad \text{Se a peca passa no segundo teste}  \\
0, \quad \text{Se a peca falha no segundo teste}
 \end{cases} 
\end{align*}
Segundo a f.m.p.\ conjunta dada por:
%Tabela ma formatada, reformular
\begin{table}[H]
  \centering
  \label{tab:x}
  \begin{tabular}{c c c c}
    $    p_{x,y} $ & $Y$& $p_x{x}$ \\
    $ X$ & 0 &1  \\
    0 & 0,1&0,2&0,3 \\
    1 & 0,2& 0,5& 0,7\\
    $p_y(y)$  & 0,3&0,7&1
  \end{tabular}
\end{table}
$P(X=0, Y=0)=0,1$, indica a proabilidade de uma peca selecionada ao acaso falhar no primeiro teste e no segundo teste.
\end{description}
\subsection{Distribuiçoes Marginais}
\begin{description}
\item [Definição:] 
  Seja $(X,Y)$ variávei aleatória bidimensional com f.m.p.\ conjunta $p_{x,y}(x,y)$. A probabilidade marginal de $X$ eh $\sum \limits_{R_{Y}} p_{x,y} (x,y)$. A probabilidade marginal de $Y$ eh $p_{y}(y) = \sum \limits_{R_{x}} p_{x,y}(x,y)$.
\item[Exemplo:] Considere o exemplo anterior, e obtenha as distribuiçoes marginais.

  Marginal de X:
  \begin{align*}
    R_{x} = \{ 0,1 \}
  \end{align*}
  \begin{align*}
    p_{x} (0) = P(X=0, Y=y)\\
    = P(X=0, Y=0) +P(X=0,Y=1)=0,1+0,2=0,3
  \end{align*}
  \begin{align*}
    p_{x} (0) = P(X=1, Y=y)\\
    = P(X=1, Y=0) +P(X=1,Y=1)=0,2+0,5=0,7
  \end{align*}
  Com isso, temos que a f.m.p.\ marginal de $X$ eh:
  \begin{table}[H]
    \centering
    \begin{tabular}{c c c}
      $x$ & $0$ &$1$ \\ \midrule
      $p_{x}$ & 0,3 & 0,7 \\
    \end{tabular}
  \end{table}
  A f.m.p.\ marginal de $Y$ eh dada por:
\begin{table}[H]
    \centering
    \begin{tabular}{c c c}
      $y$ & $0$ &$1$ \\ \midrule
      $p_{x}$ & 0,3 & 0,7 
    \end{tabular}
  \end{table}
\item [Definicao:] A v.a.\ bidimensional continua $(X,Y)$ tem f.d.p.\ conjunta dada por $f_{x,y}(x,y)$ e função acumulada $F_{x,y}(x,y)$, satisfazendo:
  \begin{enumerate}[leftmargin=*, label=\roman*., widest=IV, align=left]
    \item 
     \begin{align}
      f_{x,y} \geq 0, \quad \forall (x,y)
     \end{align} 
    \item  
     \begin{align*}
      \int \limits_{R_{y}} \int \limits_{R_{x}} f_{x,y} (x,y) dx dy = int \limits_{R_{x}} \int \limits_{R_{y}} f_{x,y} (x,y) dy dx= 1
     \end{align*} 
    \item 
      \begin{align*}
        F_{X,Y}(x,y) = P(X \leq x, Y\leq y)\\
        = F_{X,Y}(x,y) = P(X < x, Y\leq y)\\
        = F_{X,Y}(x,y) = P(X \leq x, Y< y)\\
        = F_{X,Y}(x,y) = P(X<x, Y<y)
      \end{align*}
  \end{enumerate}
\item[Propriedades]:
  \begin{enumerate}[label=(\alph*)]
    \item 
      \begin{align*}
        f_{x,y} (x,y)= \frac{\partial^2}{\partial x \partial y}F_{x,y} (x,y)
      \end{align*}
    \item 
      \begin{align*}
        F_{x,y} (x,y)= \int \limits_{- \infty}^{x} \int \limits_{- \infty}^{y} f_{x,y}(x,y) dy dx
      \end{align*}
      \item 
        \begin{align*} 
          P(a \leq X \leq b ; c \leq y \leq d) = \int \limits_{a}^{b} \int \limits_{c}^{d} f_{x,y} (x,y) dy dx
        \end{align*}
  \end{enumerate}
\item [Exemplo:] Duas caracteristicas do desempenho do motor de um foguete sao o empuxo $(X)$ e a taxa de mistura $(Y)$. Suponha que $(X,Y)$ seja uma v.a.\ continua bidimensional com f.d.p.\ conjunta dada por:
  \begin{align*}
    f_{x,y} (x,y)= \begin{cases}
      2(x+y-2xy), \quad 0\leq x \leq 1, \quad 0 \leq y \leq 1\\
      0, \quad \text{c.c.}
    \end{cases}
  \end{align*}
  Calcular a probabilidade:
  \begin{align*}
    P(0 \leq X \leq 0,5; 0,5 \leq Y \leq 1)  \\
    = \int \limits_{0}^{0.5} \int \limits_{0.5}^{1} 2(x+y -2xy)dy dx \\
    =2 \int \limits_{0}^{0.5} \left( \eval{xy}{0.5}{1} + \eval{\frac{y^2}{2}}{0.5}{1} - \eval{xy^2}{0.5}{1} \right) dx \\
    = 2 \int \limits_{0}^{0.5} \left( (\frac{1}{2} x + \frac{3}{8} - \frac{3x}{4}) \right)dx \\
    = 2 \left( \eval{\frac{x^2}{4}{0}{0.5} + \eval{\frac{3}{8}}{0}{0.5} -\eval{\frac{3x^2}{8}}{0}{0.5} \right) \\
      = \frac{5}{16}  =0.3125
  \end{align*}
\end{description}
\subsection{Densidades Marginais}
\begin{description}
  \item [Definição:] Seja $(X,Y)$ v.a.\ continua bidimensional com f.d.p.\ conjunta $f_{x,y}(x,y)$. A f.d.p.\ marginal de $X$ eh dada por $f_{x}(x)= \int \limits_{R_{y}}f_{x,y}(x,y)dy$ e a f.d.p.\ marginal dada por:
    \begin{align*}
      f_{y} (y)= \int_{R_{x}}f_{x,y} (x,y) dx
    \end{align*}
  \item [Exemplo:] No exemplo anterior, determine as distribuiçoes marginais de $X$ e $Y$. 

    \begin{align*}
      f_{x,y} (x,y) = 2(x+y -2xy), \quad x \in \[ 0,1 \] \quad y \in \[ 0,1 \]
    \end{align*}
    A marginal de $X$:
\begin{align*}
  f_{x} (x)= \int \limits_{0}^{1} 2 \left(x+y -2xy \right) dy \\ 
  = 2 \left( \eval{xy}{0}{1} + \eval{xy}{0}{1} - \eval{xy^2}{0}{1} \right)
\end{align*}
\begin{align*}
  f_{x}(x)= 2 \left( x + \frac{1}{2} - x \right)= 1
\end{align*}
Logo, a f.d.p de $X$ eh:
\begin{align*}
  f_{x} (x) = \begin{cases}
   1, \quad 0 \leq x \leq 1  \\
   0, \quad \text{c.c.}
  \end{cases}
\end{align*}
A marginal de $Y$:
\begin{align*}
  f_{y} (y) = \int \limits_{0}^{1} 2 \left(x+y -2xy\right)dx \\
  =2 \left( \eval{x^2}{0}{1}- \eval{yx^2}{0}{1}\right)\\
  = 2 \left(\frac{1}{2}+ y - y\right)=1
\end{align*}
\end{description}
\subsection{Independência Probabilística}
\begin{description}
  \item  Duas v.a.'s $X$ e $Y$ sao independentes se a sua f.m.p.\ conjunta (ou f.d.p.\ conjunta) de $(X,Y)$ for fatoravel no produto das f.m.p.\ (ou f.d.p.\ marginais )de $X$ e $Y$, isto eh:
  \begin{align}
    p_{x,y}  = p_{x}(x) \times p_{y}(y)
  \end{align}
  No caso discreto, e no caso continuo:
  \begin{align}
    f_{x,y} (x,y)= f_{x}(x) \times f_{y}(y) \quad \forall (x,y)
  \end{align}
 
\item [Exemplo:]  Suponha que uma maquina seja ajustada par auma determinada tarefa de manha e para outra tarefa a tarde. SUponha que o numero de vezes que a maquina da problema de manha e a tarde sejam representadas por variaveis  $X$ e $Y$ com f.m.p.\ conjunta dada por:
  \begin{table}[H]
    \centering
    \begin{tabular}{c c c c c}
      \toprule
      x /\ y & 0 & 1 & 2 &$p_{x}(x)$ \\ \midrule
      0 & 0,1 & 0,2 & 0,2& 0,5\\ \midrule
      1 & 0,04 & 0,08& 0,08&0,2 \\ \midrule
      2 & 0,06 & 0,12 & 0,12 & 0,3 \\ \midrule 
      $p_{Y} (y)$ & 0,2 & 0,4 &0,4 & 1 \bottomrule
      \bottomrule
    \end{tabular}
  \end{table}
  Observem que:
  \begin{align*}
    P(X=0, Y=0) = P(x=0) + P(y=0)= 0,5 \times 0,2 = 0,1
  \end{align*}
  \begin{align*}
    P(X=0, Y=1) = P(x=0) + P(y=1)= 0,5 \times 0,4 = 0,2
  \end{align*}
E assim por diante,  para todo $x$ e $y$.

Como esses produtos de probabilidades sao satisfeitos, temos que $X$ e $Y$ sao independentes.
\end{description}
\subsection{Esperanca, Covariancia, Correlacao}
\begin{description}
  \item [Definição:] Sejam $x$ e $y$ v.a.'s com $E(X)=\mu_{x}$, $E(Y)=\mu_{y}$, $Var(X)=\sigma_x^2$ e $Var(Y)=\sigma_y^2$. Seja $p_{x,y}(x,y)$ a f.m.p.\ conjunta ou $f_{x,y}(x,y)$ a f.d.p.\ conjunta de $x$ e $y$. Seja $h(x,y)$ uma função de $(X,Y)$. Definimos o valor esperado de $h(x,y)$ no caso discreto por:
    \begin{align}
      E \left(h \left(x,y\right)\right) = \sum limits_{R_{x}} \sum \limits_{R_{y}} h(x,y) \times p_{x,y}(x,y)
    \end{align}
    E no caso contínuo:
    \begin{align}
      E(h(x,y)) = \int \limits_{R_{x}} \int \limits_{R_{y}} h(x,y) f_{x,y}(x,y) dydx \\
      = \int \limits_{R_{y}} \int \limits_{R_{x}} h(x,y) f_{x,y}(x,y) dxdy
    \end{align}
    Alem disso, definimos para ambos os casos, a covariancia entre $x$ e $y$:
    \begin{align}
      \sigma_{x,y}  = Cov(X,Y) = E \left( \left(X-E(X)\right) \left(Y-E(Y)\right)\right) \\
      = E(XY) - E(X)E(Y)\\
      =E(XY)- \mu_{x}\times \mu_{y}
    \end{align}
    A correlacao entre $x$ e $y$ eh dada por:
    \begin{align}
      \rho_{x,y} = \frac{Cov(X,Y)}{\sqrt{Var(x) \times \sqrt{Var(y)}}}= \frac{\sigma_{x,y}}{\sigma_{x} \times \sigma_{y}}
    \end{align}
  \item [Exemplo:] Vamos calcular a Correlacao entre $X$ e $Y$ no exemplo anterior (exemplo da maquina).\\
    A f.m.p.\ conjunta eh:
  \begin{table}[H]
    \centering
    \begin{tabular}{c c c c c}
      \toprule
      x /\ y & 0 & 1 & 2 &$p_{x}(x)$ \\ \midrule
      0 & 0,1 & 0,2 & 0,2& 0,5\\ \midrule
      1 & 0,04 & 0,08& 0,08&0,2 \\ \midrule
      2 & 0,06 & 0,12 & 0,12 & 0,3 \\ \midrule 
      $p_{Y} (y)$ & 0,2 & 0,4 &0,4 & 1 \bottomrule
      \bottomrule
    \end{tabular}
  \end{table}
\begin{align*}
  \rho_{x,y} = \frac{Cov(x,y)}{\sigma_{x},\sigma_{y}}\\
  Cov(X,Y)=E(XY)-E(X) \times E(Y)
\end{align*}
\begin{align*}
  \mu_{x} =E(X)=\sum \limits_{R_{x}}x \times p_{x} \\
  = 0 \times 0.5 + 1 \times 0.2 + 2 \times 0.3 \\
  = 0.8
\end{align*}
\begin{align*}
  \mu_{y} =E(Y)=\sum \limits_{R_{y}}y \times p_{y} \\
  = 0 \times 0.2 + 1 \times 0.4 + 2 \times 0.4 \\
  =1.2 
\end{align*}
\begin{align*}
  E(XY) = \sum \limits_{R_{X}} \sum \limits_{R_{y}} x\times y \times p_{x,y}(x,y) \\
  = 0 \times 0 \times 0.1 + \ldots + 2 \times 2 \times 0.12\\
  = 0.96
\end{align*}
\begin{align*}
  Cov(x,y) = 0.96 - 0.8 \times 1.2 \\
  =0
\end{align*}
\begin{align*}
  \sigma_{x}^2 - \left(E(X)\right)^2
\end{align*}
Como temos:
\begin{align*}
  E(X^2) = \sum \limits_{R_{x}} x^2 \times  p_{x}
\end{align*}
Logo a $\rho_{x,y}$ eh:
\begin{align*}
  \rho_{x,y}  = \frac{0}{\sgima_{x} \times \sigma_{y}}=0 
\end{align*}
\item [Resultado]:
  \begin{description}
    \item[Independencia e correlacao]:  Se $X$ e $Y$ sao duas v.a.'s independentes, então a correlacao entre $X$ e $Y$ eh o $0$ ($\rho_{x,y}=0$).
    \item [Obs:]  A volta nao vale. Se $X$ e $Y$ sao independetes $E(XY)=E(X) \times E(Y)$.
    \end{description}
  \item [Resultados]:
    
    Para $X$ e $Y$ v.a.\ 's e $a,b \in \mathbb{R}$, constantes:
    \begin{enumerate}
      \item 
       \begin{align*}
         Var(X+Y)= Var(X)+Var(Y) + 2 Cov(X,Y) 
       \end{align*} 
     \item 
       \begin{align*}
         Var(X-Y) = Var(X) + Var(Y) -2 Cov(X,Y)
       \end{align*}
     \item Se $X$ e $Y$ sao independentes, entao:
       \begin{align*}
         Var(X \pm Y) = Var(X) +Var(Y)
       \end{align*}
     \item 
       \begin{align*}
         Var(aX \pm bY) - a^2 Var(X)+ b^2 Var(Y) \pm 2ab Cov(X,Y)
       \end{align*}
     \item Se $X$ e $Y$ sao independentes, entao:
       \begin{align*}
         Var(aX \pm bY) = a^2 Var(X) + b^2Var(Y)
       \end{align*}
    \end{enumerate}
  \item [Exemplo:] Seja $(X,Y)$ uma v.a. bidimensional cuja f.d.p.\ conjunta eh dada por:
    \begin{description}
      
    \begin{align*}
      f_{x,y} (x,y) = \begin{cases}
       9xy, \quad 0<x<y<1  \\
       0, \quad \text{c.c.}
      \end{cases}
    \end{align*}
\item [Determine:]
    \begin{enumerate}[label=(\alph*)]
      \item As marginais de $X$ e $Y$
      \item $P(0 \leq X \leq \frac{1}{2}, 0 \leq Y \leq \frac{1}{2})$
    \end{enumerate}
\item [Resolução:]
   \begin{enumerate}[label=(\alph*)]
     \item $f_{x} (x)= ?$ e $f_{y}(y)=?$
       %Grafico aqui\
      \begin{align*}
        f_{x} (x)= \int \limits_{R_{y}} f_{x,y} (x,y) dy \\
        = \int \limits_{x}^{1} 8xy dy = \eval{ \frac{8xy^2}{2} }{x}{1} \\
      = 4x(1-x^2)
      \end{align*}
      Entao, a marginal de $x$ eh:
      \begin{align*}
        f_{x} (x)= \begin{cases}
          4x(1-x^2) , \quad 0<x<1 \\
          0, \quad \text{c.c.}
        \end{cases}
      \end{align*}
        \begin{align*}
          f_{y}(y)= \int \limits_{R_{x}} f_{xy} (x,y) \\
          = \int \limits_{0}^{y} 8xy dx = \eval{\frac{8yx^2}{2}}{0}{y} = 4y^3
        \end{align*}
        Logo, a marginal de $y$ eh:
        \begin{align*}
          f_{y }(y)= \begin{cases}
           4y^3= \quad 0<y<1 \\
           0, \quad \text{c.c.}
          \end{cases}
        \end{align*}
      \item 
        \begin{align*}
          P(0 \leq X \leq \frac{1}{2}, 0 \leq Y \leq \frac{1}{2}) = \int \limits_{0}^{\frac{1}{2}} 8xy dy dx\\
          = \int \limits_{0}^{\frac{1}{2}} \int \limits_{0}^{y} 8xy dx dy \\
          = \frac{1}{16}=0.0625
        \end{align*}
   \end{enumerate} 
    \end{description}

\end{description}
